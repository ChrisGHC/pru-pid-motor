\chapter{Running the Project}

In order to run the project and hear audio from the speaker, the following steps must have been completed:

\begin{itemize}
	\item ``make'' run in pruadc1 repository directory.  Some warnings or errors may be ignored.  Check that the C code files pruadc0.c and pruadc1.c compile and firmware files are copied to /lib/firmware.
	\item USB codec is plugged into the BBG USB jack.  A speaker is plugged into the USB codec.
	\item Run command 
	
	\begin{verbatim}
	prumodout
	\end{verbatim} 
	
	at the command line.  Note that this command may cause some errors or warnings to be emitted.
	This is normal and depends on the current state of the remoteproc kernel driver.
	\item Run command 
	
	\begin{verbatim}
	prumodin
	\end{verbatim} 
	
	at the command line.
	\item Connect an analog audio source to the MCP3008 ADC input.  This is pin 1 ``CH0''.  Also connect a ground from the audio source to the analog ground pin 12 ``AGND''.  The audio source must have a positive DC offset of about one-half of 3.3 Volts.  The waveform generator of the Analog Discovery 2 is a perfect source to drive the ADC input, as it has DC and AC magnitude settings in the GUI.  A low audio frequency is recommended.  A nice pleasant tone is 440 Hz (the musical note A).
	\item  Finally, the user-space program is ready to be started!
	
	\begin{verbatim}
	cd user_space
	./fork_pcm_pru
	\end{verbatim}
\end{itemize}

If all goes well, a tone should be emitted from the speaker.  The program should indicate that a command to ALSA aplay is running in one of the processes:

\begin{verbatim}
Playing raw data 'soundfifo' : Signed 16 bit Little Endian, Rate 8000 Hz, Mono
\end{verbatim}

Stopping the program will require two Ctrl-C commands in series.

The PRU firmwares will continue to run.  To stop them, issue the command

\begin{verbatim}
prumodout
\end{verbatim}

at the command line, and the PRUs will be halted.
