\chapter{Running the Project}

It is assumed the numerous steps described in prior chapters have been completed to enable the RemoteProc framework drivers and configure the BBG for compiling PRU C code.  The Device Tree must have been successfully edited and re-compiled and installed, and the shell scripts prumodin and prumodout have been copied to /usr/bin.

In order to run the project and successfully control the motor, follow these steps:

\begin{itemize}
	\item run ``make'' in the software repository directory.  Some warnings or errors may be ignored.  Check that the C code files prumsg.c, PRU\_PID\_0.c and PRU\_IO\_1.c compile and firmware files am335x-pru0-fw and am335x-pru1-fw are copied to /lib/firmware.  The user-space binary prumsg should be copied to /usr/bin.
	
	\item Run command 
	
	\begin{verbatim}
	prumodin
	\end{verbatim} 
	
	at the command line.  This command should start firmware execution in the PRUs.
	If all goes well, the motor should begin turning with a default setpoint value of 3000.
	
	\item  Finally, the user-space program can be used.
	
	\begin{verbatim}
	sudo ./prumsg s 4000
	\end{verbatim}
	
	The above command changes the setpoint to 4000 rpm.  The motor speed should increase.
\end{itemize}

The PRU firmwares will continue to run.  To stop them, issue the commands

\begin{verbatim}
sudo prumodout
sudo rmmod pru_rproc
sudo rmmod pruss
\end{verbatim}

at the command line, and the PRUs will be halted.  The motor will stop.

\section{User-space Program prumsg Command Listing}

The user-space executable file prumsg is capable of several control and monitoring functions.  These commands are issued from a shell and a complete listing of the possible commands is listed below.

\begin{longtable}{ll}
\caption{Commands of User-space Program prumsg}\\
\toprule
Example command & Command function \\\midrule
prumsg 30 s 5000 & Set setpoint (RPM) \\ 
prumsg 30 p 300 & Set Kp, proportional feedback coefficient \\ 
prumsg 30 i 300 & Set Ki, integral feedback coefficient \\ 
prumsg 30 d 300 & Set Kd, derivative feedback coefficient \\ 
prumsg 30 o 3000 & Set output PWM duty cycle (see note below) \\ 
prumsg 30 rs & Readback setpoint (RPM) \\ 
prumsg 30 rp & Readback Kp \\ 
prumsg 30 ri & Readback Ki \\ 
prumsg 30 rd & Readback Kd \\ 
prumsg 30 re & Readback encoder RPM \\
prumsg 30 ro & Readback output PWM
\end{longtable}

Notes on the above:
\begin{enumerate}
\item If not operating as root, ``sudo'' will be required.
 
\item ``30'' in the table above refers to the character device /dev/rpmsg\_pru30.  ``31'' can also be used, as this character device is also established between PRU1 and user-space.
 
\item The example for setting the PWM (prumsg 30 o 3000) will not have effect with the PID controller running.  However, this command is useful for debugging purposes.  If the PID controller is not running in PRU0, then the command will work and the PWM output will change.  The simplest way to do this is to remove firmware am335x-pru0-fw from directory /lib/firmware.  Reboot and restart the system.  PRU1 will still function in system control mode, but the PID calculations will not be performend by PRU0 and the system will operate in ``open-loop'' mode.  This is excellent for checking the PWM output to the motor driver IC and the motor connections.  The motor should properly respond to changes in the PWM duty cycle by issuing the prumsg 30 s (pwm value) command.
\end{enumerate}

\section{Server and GUI Interface from TI Project}

The TI project includes a very clever PHP web page and server interface.  This was found to be mostly functional, and the server shell script and web page implementation is included in the pru\_pid\_server directory of the Git repository.

The server is easy to run.  Simply copy the contents of the pru\_pid\_server directory to /var/www/html which should already be included in the Beaglebone Green IOT image.

Now, using a browser on the local network, browse to this URL:

\begin{verbatim}
10.0.0.2:8080
\end{verbatim}

In the above example, the BBG is set to a static IP of 10.0.0.2.

This was found to partially function, the graphics successfully updates, however, the capability to update the setpoint and PID parameters did not function.

This project does not support this function.  Since the Beagleboard project is heavily invested in node.js and ``Bonescript'', it would probably make sense to change this function from PHP/html to a node-based web interface and use a web-socket for data exchange and parameter control.  This feature may be added in the future to this project.
