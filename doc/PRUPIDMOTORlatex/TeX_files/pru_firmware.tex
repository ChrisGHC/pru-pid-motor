%    Documentation for PRU ADC Project
%    Copyright (C) 2016  Gregory Raven
%
%    This program is free software: you can redistribute it and/or modify
%    it under the terms of the GNU General Public License as published by
%    the Free Software Foundation, either version 3 of the License, or
%    (at your option) any later version.
%
%    This program is distributed in the hope that it will be useful,
%    but WITHOUT ANY WARRANTY; without even the implied warranty of
%    MERCHANTABILITY or FITNESS FOR A PARTICULAR PURPOSE.  See the
%    GNU General Public License for more details.
%
%    You should have received a copy of the GNU General Public License
%    along with this program.  If not, see <http://www.gnu.org/licenses/>.

\chapter{PRU Firmware and User-space Program}

The ``PRU Firmware'' are two binary files which are placed in the directory /lib/firmware.
These files must have specific names as follows:

\begin{itemize}
	\item am335x-pru0-fw
	\item am335x-pru1-fw
\end{itemize}

The Makefile includes cp commands to copy the firmwares to the /lib/firmware directory.

\section{The PID Firmware in PRU0:  Digital Feedback Loop (pru0adc.c)}

The SPI bus C program roughly follows the PRU assembly code written by Derek Molloy.  The C code is compiled to a binary file am335x-pru0-fw. The firmware is loaded into PRU0 automatically by the Remoteproc kernel driver.

The program begins with code sequestered from an example code file in TI's PRU Software
Support Package.  This codes establishes the character device driver via the ``Remote Proc Messenger''
kernel driver.  There are several lines of RPMsg ``set-up'' code which appear at the top of the file.

Here is the path to the file from the root directory of the PRU Software Support Package:

\begin{verbatim}
pru-software-support-package/examples/am335x/PRU_RPMsg_Echo_Interrupt0
\end{verbatim}

There is a sort of ``priming'' process required whereby a user space program writes
to the device driver.  The initializes the character driver, which allows it to write data from
the PRU to the character driver and thus making the data available in user-space.  This is the critical
code which performs this function:

\begin{verbatim}
  //  This section of code blocks until a message is received from ARM.
  while (pru_rpmsg_receive(&transport, &src, &dst, payload, &len) !=
  PRU_RPMSG_SUCCESS) {
  }
\end{verbatim}

This empty while-loop continues until the user-space code writes a message to the character driver.  Upon receipt of a message, the data transport channel is ready to go, and the program breaks out of the while-loop.

After the initialization is complete, the program enters a for-loop.  The SPI bus is implemented inside this for-loop.  This is done by ``bit-twiddling'' of register \_\_R30, which is 32 bits in length.  The individual bits in \_\_R30, in turn determine the ``high'' or ``low'' state at the GPIO, and thus the header pins.  The GPIO multiplexing is set via the ``Universal IO'', which is described in a later chapter.

The code utilizes timing delays and sequential setting and unsetting of bit values of \_\_R30.  This is done per the requirements shown in the MCP3008 ADC data sheet. Each pass through the for-loop configures the ADC, and then captures a single 10-bit sample from the ADC.

The top of the for-loop is blocked by a while ``polling'' loop.  The operand of the while-loop is the value of a PRU shared memory location.  The PRU1 timing clock code writes to this location at precisely timed intervals.  When the while loop detects that three of the bits change from 0s to 1s, the while loop is broken and the SPI bus data acquisition sequence begins.  This action is what determines the 8 kHz data sampling rate.

The samples are accumulated in a buffer (int16\_t payload[256]).  When the buffer is filled, the data is written to the character device via a function provided by the RPMsg kernel driver.  Here is the code:

\begin{verbatim}
    //  Send frames of 245 samples.
    //  The entire buffer size of 512 can't be used for
    //  data.  Some space is required by the "header".
    //  The data is offset by 512 and then multiplied
    //  to make appropriately scaled 16 bit signed integers.
    payload[dataCounter] = 50 * ((int16_t)data - 512);
    dataCounter = dataCounter + 1;
    
    if (dataCounter == 245) {
    pru_rpmsg_send(&transport, dst, src, payload, 490);
    dataCounter = 0;
    }
\end{verbatim}

The dataCounter variable is incremented at the end of each pass through the for loop.  When the dataCounter hits 245, the pru\_rpmsg\_send command is used to write data to the character device. The dataCounter variable is reassigned to zero and the process repeats.

The maximum buffer size which can be handled by RPMsg is 512 bytes (or 256 16 bit integers).  However, not all of the buffer can be used as there is a ``header'' included which takes a few bytes.  The number of samples transmitted, 245, was determined empirically.  Some study of the kernel drivers needs to be accomplished to better understand this limitation.

Timings are critical, and this was accomplished by using the compiler intrinsic \_\_delay\_cycles.  Each delay is an absolute value of 5 nanoseconds.  This scheme has sufficient timing precision to implement the SPI bus in real time.  

\section{The Firmware in PRU1: PID Control (pru1adc.c)}



